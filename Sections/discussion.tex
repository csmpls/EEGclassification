\section{Discussion}

We find that logarithmic binning dramatically decreases the computational expense of EEG-based calibration and classification without a significant detriment to accuracy. 

Logarithmic binning could enable co-adaptive, online BCI with as few as one dry EEG sensor, making online calibration much more performant on mobile or embedded processors with limited computational resources. Alternatively, since logarithmic dramatically decreases the size of data fed to the classification algorithm, the technique could allow calibration to occur ``in the cloud" - the BCI could pre-process the data on board, bin it, and ship this data to a more powerful server, which could process it online. By some combination of cloud-based and on-board processing, BCIs could gain from the accuracy of computationally expensive analytics without having to perform these computations on-board.

\textcolor{red}{\bf [On on-board versus cloud, you want to outline which part of computations should be on-board, and which part should be in the cloud and why. If you go in this direction, you might also want to justify your design, which requires benchmarking each (sub-)task. My impression is that it's a little beyond this paper as we have not even an acquisition/processing app up and running.]}
