
\section{Abstract}

Using adaptive, machine learning (ML)-based classifiers, electroencephalography (EEG) signals have been leveraged to build applications ranging from mind-controlled keyboards to prosthetic arms and hands. Unfortunately these applications require elaborate scanning caps. The data they produce are large in memory and computationally expensive to process. Can we achieve acceptable classifier performance using compressed EEG data and minimal scanning resolution? In this study, we examine tradeoffs between compression, classifier accuracy and classifier performance using mental task data recorded from a single electrode. ``by compressing data to 1/x their original size using a logarithmic binning method, we find an \ital{increase} in accuracy from a to b, and fail to find evidence of significant degredation in performance at up to 1/y of the data's original size.''  ``this compression is met with a 10^m increase in classifier speed over uncompressed data.''  Our findings suggest that compression techniques could enable ML-based EEG in the cloud, as compression minimizes data throughput, or on lightweight, embeded processors.