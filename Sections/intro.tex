\section{Introduction}

Brain-computer interface (BCI) refers to the generation of machine-interpretable signals by brain signals, unmediated by musclar or nervous activity. BCI systems aim to recognize a user's mental cues so that the user's intentions can be executed (or transmitted) electronically as discrete symbols - a task often conceptualized as a pattern recognition problem []. Its difficulty stems from the non-stationary nature of mind. The mental representations we wish to classify are constantly changing. In order to compensate for the variability of brain-based signals, recent work has leveraged adaptive, machine learning approaches for classifying brainscan data, and early successes with this technique have yielded brain-controlled keyboards [hex-o-spell], wheelchairs [millan], and prosthetic arms and hands [tobi].

However, these applications require high electrode density and high temporal resolution, creating dense and high-dimensional feature vectors. This makes classification systems computationally expensive.[] Classification bottlenecks threaten the responsiveness of BCI applications to their users, and complex scanning equipment places high requirements on end users, and on their hardware. 

Do BCI applications really require the dense electrode arrays and high scanning resolutions that past work has traditionally used? Is it possible to build a working BCI system with compressed feature vectors and a minimal number of sensors? In this study, we seek to examine fundamnetal tradeoffs between compression, classifier accuracy and online classifier performance for ``in-the-wild'' BCI systems. We use recordings from a single, dry eleectroencephalographic (EEG) electrode, and present a calibration tool that finds, for each user, the optimal compromise between classification accuracy and the computational expense of the classifier. By modelling how vector compression affects both the accuracy of the BCI system and the computational performance of the classifier, we hope to shed light on the optimal compromise between usability (accuracy and training time) and ease of deployment (computational expense and sensor resolution) in real-world BCI systems.



