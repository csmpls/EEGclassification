\section{Introduction}


Brain-computer interface (BCI) systems establish a direct communcative link between the brain and an electronic system \cite{dornhege_toward_2007,mcfarland_brain-computer_2011}.  Recently, the combination of machine learning algorithms and non-invasive electroencephelographs (EEG) has yielded proof-of-concept systems ranging from brain-controlled keyboards and wheelchairs to prosthetic arms and hands \cite{blankertz_note_2007,millan_combining_2010,d._mattia_brain_2011,hill_practical_2014,campbell_neurophone:_2010}. 
% TODO: more cites on cool EEG applications?

There are a few reasons why these systems have not been widely adopted outside of lab settings. For one, they require large, complex scanning caps, which are impractical for disabled users and generally undesirable for ergonomic reasons \cite{ekandem_evaluating_2012,leeb_transferring_2013}. Meanwhile, the amount of data produced by these caps is large, requires significant processing, and are unsuited to mobile computing environments, in which these systems will most likely be deployed (e.g., smartphones, watches, embedded systems in scanning devices, etc). Finally, BCIs often require upward of an hour to calibrate to their users, and may require regular recalibration due to the nonstationary nature of EEG signals. \cite{vidaurre_fully_2006, vidaurre_co-adaptive_2011,blankertz_non-invasive_2007}

For BCI systems to enjoy wider adoption ``in the wild,'' they must calibrate to individual users quickly and achieve decent information transfer rates (ITR), but with fewer sensors than their lab-based counterparts, and with noisier signals, as data acquisition will occur while people are performing daily tasks, moving, walking, talking, a}nd so on. As an added challenge, their computational firepower may be limited by the mobile \& wearable computing architectures on which they will must likely be deployed.

Do we really need dense, high-dimensional EEG data to achieve acceptable accuracy? In this study, we use recordings from a consumer-grade single, dry electroencephalographic (EEG) sensor to simulate the calibration of a simple BCI, and investigate the effect of a novel signal extraction technique on the system's computational performance and accuracy. \textcolor{red}{\bf [shortly summarize its underpinning]}. First, we find that our signal extraction technique significantly increases the computational speed of a classification-based BCI without a significant detriment to accuracy. Second, we find evidence that this technique can be used to build effective mental task classifiers with under a minute of training data.

