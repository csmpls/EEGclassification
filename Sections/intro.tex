\section{Introduction}


Brain-computer interface (BCI) systems establish a direct communcative link between the brain and an electronic system. \cite{dornhege_toward_2007,mcfarland_brain-computer_2011} While early systmes required 
%TODO:blurb here????
[[[ electrodes implanted in the user's s ]]], newer work has used non-invasive electroencephelographs (EEG) to build proof-of-concept BCI systems ranging from brain-controlled keyboards [hex-o-spell] and wheelchairs [millan] to prosthetic arms and hands [tobi] \cite{blankertz_note_2007,millan_combining_2010,d._mattia_brain_2011}

There are a few reasons why these systems have not been widely deployed outside of lab settings. \cite{leeb_transferring_2013,hill_practical_2014} For one, they require large, complex scanning caps, which are impractical for disabled users and generally undesirable for ergonomic reasons. \cite{ekandem_evaluating_2012} Meanwhile, the data these caps produce is large, which could be computationally unwieldy on the mobile computing environments in which these systems will most likely be deployed (eg., smartphones, watches, embedded systems in scanning devices, etc). 

For BCI systems to find wide adoption ``in the wild,'' they must calibrate to individual users quickly, acheive decent information transfer rates (ITR), and must run on mobile & wearable computing architectures. Since naturalistic environments will also introduce interference in the signal, they must also cope with lower signal quality than their lab-based counterparts and, for aesthetic and ergonomic reasons, must remain small, and ought not to require special placement or the application of gels. As a result, BCI systems must squeeze more information out of less raw data: they should approach in-lab metrics of efficacy with fewer and more comfortable sensors. 

Can a small number of inexpensive, low-quality sensors acheive acceptable accuracy after a brief, online calibration phase? In this study, we use recordings from a single, dry electroencephalographic (EEG) sensor to interrogate the efficacy of a novel signal extraction technique based on the logarithmic binning of power spectra over time.

 % TODO: To wit, we find...............................................................................................................................................................................................................................................


\bibliographystyle{plain}
\bibliography{references}