
\section{Machine learning approaches in EEG-based BCI}
Electroencephalography (EEG) is a popular mechanism for acheiving BCI, as its sensors are inexpensive, portable and non-invasive. However, EEG has limited spatial resolution, high variability between subjects and trials, and the ``dry'' electrodes suitable for everyday use yield poor signal quality. 

...it is important that the predictions of BCI systems be verifiable in real-time [lifshitz]

\section{Performance challenges in online, adaptive BCI}

\section{Compressing feature vectors for performant, online classification}
speed is important at training as well as testing time.................decrease vector size==>increase

Although many studies approached performance issues in online BCI systems,  few have systematically examined how BCI applications could leverage compression to increase the performance of ML-base classification.